\usepackage[utf8]{inputenc} %кодировка
\usepackage[T1,T2A]{fontenc} % тоже кодировка
\usepackage[english,russian]{babel} %языки

\newcommand\Decide[1]{#1}
\makeatletter
\def\sectionsuffix      {}
\def\subsectionsuffix   {\quad}
\def\subsubsectionsuffix{\quad}
\def\paragraphsuffix    {\quad}
\renewcommand\@seccntformat[1]{\csname the#1\endcsname\csname#1suffix\endcsname}
\renewcommand\thesection{\protect\Decide{\@arabic\c@section}}
\renewcommand\thesubsection{\@arabic\c@section.\@arabic\c@subsection}
\makeatother
\renewcommand\Decide[1]{}

\usepackage{multirow}
\usepackage{tabularx,booktabs}

\usepackage{ulem}

\usepackage{graphicx, float} %пакет для единого оформления всех плавающих объектов (избегаем повторяющихся команд в документе)

\usepackage{pgfplots}
\pgfplotsset{compat=1.18, width=10cm}

\usepackage{tikz}

\DeclareGraphicsExtensions{.pdf,.png,.jpg,.eps}%форматы 
\usepackage{titlesec}
\setcounter{secnumdepth}{4}
\titleformat{\paragraph}
{\normalfont\normalsize\bfseries}{\theparagraph}{1em}{}
\titlespacing*{\paragraph}
{0pt}{3.25ex plus 1ex minus .2ex}{1.5ex plus .2ex}


\graphicspath{{images/}} % выбираем папку, в которую сохраняем все рисунки, чтобы не было хаоса файлов

\usepackage{amsmath,amssymb}%математические формулы и символы
\usepackage{amsthm}

\usepackage[a4paper,left=20mm,right=20mm,top=20mm,bottom=20mm]{geometry} % устанавливает поля документа

\parindent=5ex %красная строка
\parskip=1mm %расстояние между параграфами
\linespread{1.5}
\usepackage{indentfirst} %делать отступ в начале параграфа

\usepackage{hyperref} % добавление ссылок

\def\hmath$#1${\texorpdfstring{{\rmfamily\textit{#1}}}{#1}}
%настройка подписей плавающих объектов

\makeindex %нумерация 

\usepackage{array,graphicx,caption} %картинки, подписи, таблицы
%\usepackage{endfloat} - для вывода картинок со списком в конце файла

\usepackage{caption} %подписи к картинкам

\usepackage[labelformat=simple]{subcaption} %для subfigure
\renewcommand\thesubfigure{(\alph{subfigure})}

\usepackage[export]{adjustbox} %чтобы влево-вправо картинки ставить

\usepackage[labelformat=simple]{subcaption}
% метка subfigure: "(а)" вместо дефолтного "а"

\renewcommand\thesubfigure{(\alph{subfigure})} % для продвинутого captionof

\usepackage{afterpage,placeins} % для барьеров 

\usepackage{wrapfig} %добавление wrapfig

\usepackage[nottoc]{tocbibind} %подключает в содержание список лит-ры

\usepackage{multicol}

\usepackage{listings}
\usepackage{etoolbox}
\usepackage{color}

\lstset{
    inputencoding=utf8,      % Кодировка исходного кода (вашего файла .tex или включаемого файла)
    extendedchars=\true,     % Разрешает использование не-ASCII символов. Критически важно!
    language=C               % Укажите ваш язык программирования
}

% Настройка listings для JavaScript/TypeScript
\lstdefinelanguage{JavaScript}{
  keywords={typeof, new, true, false, catch, function, return, null, catch, switch, var, if, in, while, do, else, case, break, const, let, async, await, export, import, from, as, default},
  keywordstyle=\color{blue}\bfseries,
  ndkeywords={class, export, boolean, throw, implements, import, this},
  ndkeywordstyle=\color{darkgray}\bfseries,
  identifierstyle=\color{black},
  sensitive=false,
  comment=[l]{//},
  escapeinside={|*}{*|},
  morecomment=[s]{/*}{*/},
  commentstyle=\color{purple}\ttfamily,
  stringstyle=\color{red}\ttfamily,
  morestring=[b]',
  morestring=[b]"
}

% ===== НАЧАЛО: Определение языка JSON =====
\usepackage{xcolor} % Нужен для подсветки цветом

\colorlet{punct}{red!60!black}
\colorlet{numb}{magenta!60!black}
\colorlet{delim}{blue!60!black}
\definecolor{backcolour}{rgb}{0.95,0.95,0.92}

\lstdefinelanguage{json}{
    basicstyle=\small\ttfamily, % Ваш стиль
    backgroundcolor=\color{backcolour},
    showstringspaces=false,
    breaklines=true,            % Перенос длинных строк
    literate=
     *{0}{{{\color{numb}0}}}{1}
      {1}{{{\color{numb}1}}}{1}
      {2}{{{\color{numb}2}}}{1}
      {3}{{{\color{numb}3}}}{1}
      {4}{{{\color{numb}4}}}{1}
      {5}{{{\color{numb}5}}}{1}
      {6}{{{\color{numb}6}}}{1}
      {7}{{{\color{numb}7}}}{1}
      {8}{{{\color{numb}8}}}{1}
      {9}{{{\color{numb}9}}}{1}
      {:}{{{\color{punct}{:}}}}{1}
      {,}{{{\color{punct}{,}}}}{1}
      {\{}{{{\color{delim}{\{}}}}{1}
      {\}}{{{\color{delim}{\}}}}}{1}
      {[}{{{\color{delim}{[}}}}{1}
      {]}{{{\color{delim}{]}}}}{1},
}
% ===== КОНЕЦ: Определение языка JSON =====

\usepackage{enumitem}
