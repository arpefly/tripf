\documentclass[unicode,a4paper,14pt]{extarticle}

\usepackage[utf8]{inputenc} %кодировка
\usepackage[T1,T2A]{fontenc} % тоже кодировка
\usepackage[english,russian]{babel} %языки

\newcommand\Decide[1]{#1}
\makeatletter
\def\sectionsuffix      {}
\def\subsectionsuffix   {\quad}
\def\subsubsectionsuffix{\quad}
\def\paragraphsuffix    {\quad}
\renewcommand\@seccntformat[1]{\csname the#1\endcsname\csname#1suffix\endcsname}
\renewcommand\thesection{\protect\Decide{\@arabic\c@section}}
\renewcommand\thesubsection{\@arabic\c@section.\@arabic\c@subsection}
\makeatother
\renewcommand\Decide[1]{}

\usepackage{multirow}
\usepackage{tabularx,booktabs}

\usepackage{ulem}

\usepackage{graphicx, float} %пакет для единого оформления всех плавающих объектов (избегаем повторяющихся команд в документе)

\usepackage{pgfplots}
\pgfplotsset{compat=1.18, width=10cm}

\usepackage{tikz}

\DeclareGraphicsExtensions{.pdf,.png,.jpg,.eps}%форматы 
\usepackage{titlesec}
\setcounter{secnumdepth}{4}
\titleformat{\paragraph}
{\normalfont\normalsize\bfseries}{\theparagraph}{1em}{}
\titlespacing*{\paragraph}
{0pt}{3.25ex plus 1ex minus .2ex}{1.5ex plus .2ex}


\graphicspath{{images/}} % выбираем папку, в которую сохраняем все рисунки, чтобы не было хаоса файлов

\usepackage{amsmath,amssymb}%математические формулы и символы
\usepackage{amsthm}

\usepackage[a4paper,left=20mm,right=20mm,top=20mm,bottom=20mm]{geometry} % устанавливает поля документа

\parindent=5ex %красная строка
\parskip=1mm %расстояние между параграфами
\linespread{1.5}
\usepackage{indentfirst} %делать отступ в начале параграфа

\usepackage{hyperref} % добавление ссылок

\def\hmath$#1${\texorpdfstring{{\rmfamily\textit{#1}}}{#1}}
%настройка подписей плавающих объектов

\makeindex %нумерация 

\usepackage{array,graphicx,caption} %картинки, подписи, таблицы
%\usepackage{endfloat} - для вывода картинок со списком в конце файла

\usepackage{caption} %подписи к картинкам

\usepackage[labelformat=simple]{subcaption} %для subfigure
\renewcommand\thesubfigure{(\alph{subfigure})}

\usepackage[export]{adjustbox} %чтобы влево-вправо картинки ставить

\usepackage[labelformat=simple]{subcaption}
% метка subfigure: "(а)" вместо дефолтного "а"

\renewcommand\thesubfigure{(\alph{subfigure})} % для продвинутого captionof

\usepackage{afterpage,placeins} % для барьеров 

\usepackage{wrapfig} %добавление wrapfig

\usepackage[nottoc]{tocbibind} %подключает в содержание список лит-ры

\usepackage{multicol}

\usepackage{listings}
\usepackage{etoolbox}
\usepackage{color}

\lstset{
    inputencoding=utf8,      % Кодировка исходного кода (вашего файла .tex или включаемого файла)
    extendedchars=\true,     % Разрешает использование не-ASCII символов. Критически важно!
    language=C               % Укажите ваш язык программирования
}

% Настройка listings для JavaScript/TypeScript
\lstdefinelanguage{JavaScript}{
  keywords={typeof, new, true, false, catch, function, return, null, catch, switch, var, if, in, while, do, else, case, break, const, let, async, await, export, import, from, as, default},
  keywordstyle=\color{blue}\bfseries,
  ndkeywords={class, export, boolean, throw, implements, import, this},
  ndkeywordstyle=\color{darkgray}\bfseries,
  identifierstyle=\color{black},
  sensitive=false,
  comment=[l]{//},
  escapeinside={|*}{*|},
  morecomment=[s]{/*}{*/},
  commentstyle=\color{purple}\ttfamily,
  stringstyle=\color{red}\ttfamily,
  morestring=[b]',
  morestring=[b]"
}

% ===== НАЧАЛО: Определение языка JSON =====
\usepackage{xcolor} % Нужен для подсветки цветом

\colorlet{punct}{red!60!black}
\colorlet{numb}{magenta!60!black}
\colorlet{delim}{blue!60!black}
\definecolor{backcolour}{rgb}{0.95,0.95,0.92}

\lstdefinelanguage{json}{
    basicstyle=\small\ttfamily, % Ваш стиль
    backgroundcolor=\color{backcolour},
    showstringspaces=false,
    breaklines=true,            % Перенос длинных строк
    literate=
     *{0}{{{\color{numb}0}}}{1}
      {1}{{{\color{numb}1}}}{1}
      {2}{{{\color{numb}2}}}{1}
      {3}{{{\color{numb}3}}}{1}
      {4}{{{\color{numb}4}}}{1}
      {5}{{{\color{numb}5}}}{1}
      {6}{{{\color{numb}6}}}{1}
      {7}{{{\color{numb}7}}}{1}
      {8}{{{\color{numb}8}}}{1}
      {9}{{{\color{numb}9}}}{1}
      {:}{{{\color{punct}{:}}}}{1}
      {,}{{{\color{punct}{,}}}}{1}
      {\{}{{{\color{delim}{\{}}}}{1}
      {\}}{{{\color{delim}{\}}}}}{1}
      {[}{{{\color{delim}{[}}}}{1}
      {]}{{{\color{delim}{]}}}}{1},
}
% ===== КОНЕЦ: Определение языка JSON =====

\usepackage{enumitem}


\begin{document}
\setcounter{page}{3}

\tableofcontents
\newpage

\section{Введение}

Организация совместных поездок и мероприятий с друзьями и знакомыми стала обыденным явлением. При этом одной из наиболее трудоёмких задач является справедливое разделение общих расходов между участниками. В процессе путешествия или группового мероприятия возникают различные траты: проживание, питание, транспорт, развлечения. Зачастую один человек оплачивает расходы за всю группу, а затем возникает необходимость вести учёт и производить взаиморасчёты.

Традиционные методы ведения учёта расходов -- использование бумажных блокнотов, таблиц в мессенджерах или Google-таблицах -- создают ряд проблем:
\begin{itemize}
    \item Высокая вероятность ошибок при ручном подсчёте
    \item Сложность отслеживания изменений и истории операций
    \item Отсутствие автоматического расчёта итоговых долгов
    \item Необходимость дублирования информации между участниками
\end{itemize}

Существующие решения для разделения расходов, такие как Splitwise, Tricount и Settle Up, решают эти проблемы, хоть и имеют свои недостатки.

\newpage

\section{Что хотим получить}

Результатом этой курсовой работы должно стать веб-приложение для управления групповыми расходами. 
Назовём это приложение \textbf{TripF} (Trip Finance).

Основные требования к системе: 

\begin{itemize}
    \item Интуитивно понятный интерфейс;
    \item Регистрация пользователей;
    \item Создание групп;
    \item Возможность приглашения других пользователей в группу;
    \item Добавление расходов;
    \item Автоматический расчёт долгов;
    \item Отметка о выполненных платежах;
    \item Возможность удалять/редактировать расходы, удалять участников из группы, удалять группы.
\end{itemize}

\section{Средства реализации}
\subsection{Клиентская часть}

Для разработки клиентской части приложения использовались следующие технологии:

\textbf{React 19} \cite{react} -- современная JavaScript-библиотека для создания пользовательских интерфейсов. 

\textbf{Next.js 16} \cite{nextjs} -- React-фреймворк с поддержкой серверного рендеринга. Используется для: API Routes для создания backend-функционала, file-based роутинга для удобной навигации.

\textbf{TypeScript 5} \cite{typescript} -- типизированный JavaScript. Обеспечивает: Статическую проверку типов на этапе разработки, улучшение читаемости и поддерживаемости кода.

\textbf{Tailwind CSS 4} \cite{tailwindcss} -- utility-first Iinline CSS фреймворк. Позволяет определять стили прямо в HTML-коде, используя предопределённые классы-хелперы.

\textbf{SWR 2.3} \cite{swr} -- библиотека для управления состоянием и кэширования данных. Используется для автоматического кэширования ответов API (то есть позволяет избежать повторных запросов к серверу).

\subsection{Серверная часть}

\textbf{Node.js} -- серверная платформа на базе V8 JavaScript-движка. Обеспечивает единый язык для фронта и бэка (TypeScript) и большую экосистему пакетов npm.

\textbf{Next.js API Routes} -- встроенная в Next.js возможность создания API-эндпоинтов (файлы в папке \texttt{app/api/} автоматически становятся API-роутами).

\textbf{Better-SQLite3 12.4} \cite{bettersqlite3} -- библиотека для работы с SQLite \cite{sqlite}. Выбор обусловлен простым синтаксисом и высокой скоростью работы.

\textbf{Custom HTTPS Server} -- собственный сервер на базе Node.js \texttt{https} модуля \cite{nodejs-https} (\texttt{server.js}). Так как у меня iPhone он отказывается открывать http сайты из локальной сети. Поэтому были созданы самоподписанные сертификаты и соотвественно для этого нужен такой <<костыль>>.

\subsection{Обоснование выбора технологий}

Главным критерием выбора такого стека -- единая экосистема Next.js. Это позволяет использовать один и тот же язык для фронта и бэка, что упрощает разработку и позволяет переиспользовать код (типы данных, утилиты).

\newpage

\section{Архитектура системы}
\subsection{Общая архитектура приложения}

Приложение TripF построено по классической клиент-серверной архитектуре с использованием паттерна MVC (Model-View-Controller), адаптированного для современных веб-приложений.

\textbf{Архитектурные слои}:

\begin{enumerate}
    \item \textbf{Слой представления (View)} -- React-компоненты, отвечающие за отображение интерфейса и взаимодействие с пользователем (\texttt{app/components/}).
    \item \textbf{Слой маршрутизации (Controller)} -- Next.js страницы \newline (\texttt{app/[route]/page.tsx}) и API Routes (\texttt{app/api/[route]/route.ts}).
    \item \textbf{Бизнес-логика} -- функции для работы с данными, расчётов, валидации (\texttt{app/lib/}).    
    \item \textbf{Слой данных (Model)} -- взаимодействие с SQLite через библиотеку Better-SQLite3. Схема БД и функции доступа в \texttt{app/lib/db.ts}.
\end{enumerate}

\textbf{Поток данных}:
\begin{enumerate}
    \item Пользователь взаимодействует с React-компонентом
    \item Компонент вызывает API-эндпоинт через fetch/SWR
    \item API Route проверяет авторизацию, валидирует данные
    \item Бизнес-логика обрабатывает запрос, работает с БД
    \item Результат возвращается в виде JSON
    \item SWR кэширует ответ и обновляет компонент
    \item React перерисовывает интерфейс
\end{enumerate}

\subsection{Структура базы данных}
База данных состоит из 8 таблиц.

\begin{figure}[h]
    \centering
    \includegraphics[width=\textwidth]{images/db_scheme.png}
    \caption{Схема базы данных}
    \label{fig:db_scheme}
\end{figure}

Основные сущности: users -- таблица с пользователями, groups -- таблица с группами, expenses -- таблица с расходами, settlement\_payments -- таблица с платежами.

Вспомогательные сущности: group\_members, expense\_splits -- для связи многие ко многим, group\_invites -- таблица с приглашениями в группу, sessions -- таблица с сессиями пользователей.

\subsection{API-routers}
\textbf{Аутентификация} (\texttt{/api/auth/*}):
\begin{itemize}
    \item \texttt{POST /api/auth/register} -- регистрация нового пользователя. Создаёт пользователя, хеширует пароль, создаёт сессию.
    
    \item \texttt{POST /api/auth/login} -- вход в систему. Проверяет пароль, создаёт сессию, устанавливает cookie.
    
    \item \texttt{POST /api/auth/logout} -- выход из системы. Удаляет сессию и cookie.
    
    \item \texttt{GET /api/auth/me} -- получение информации о текущем пользователе.
\end{itemize}

\textbf{Управление группами} (\texttt{/api/groups/*}):
\begin{itemize}
    \item \texttt{GET /api/groups} -- получение списка групп текущего пользователя
    
    \item \texttt{POST /api/groups} -- создание новой группы.
    \item 
    \item \texttt{GET /api/groups/[id]} -- получение информации о группе с участниками
    
    \item \texttt{DELETE /api/groups/[id]} -- удаление группы.
\end{itemize}

\textbf{Расходы} (\texttt{/api/groups/[id]/expenses/*}):
\begin{itemize}
    \item \texttt{GET /api/groups/[id]/expenses} -- список всех расходов группы с разбивкой
    
    \item \texttt{POST /api/groups/[id]/expenses} -- создание расхода.  После создания генерируется событие \texttt{expense:created}.
    
    \item \texttt{GET /api/groups/[id]/expenses/[expenseId]} -- получение конкретного расхода
    
    \item \texttt{PUT /api/groups/[id]/expenses/[expenseId]} -- редактирование расхода. Удаляет старые splits и создаёт новые.
    
    \item \texttt{DELETE /api/groups/[id]/expenses/[expenseId]} -- удаление расхода. Каскадно удаляет splits благодаря FOREIGN KEY.
\end{itemize}

\textbf{Платежи} (\texttt{/api/groups/[id]/payments/*}):
\begin{itemize}
    \item \texttt{GET /api/groups/[id]/payments} -- список всех платежей группы
    
    \item \texttt{POST /api/groups/[id]/payments} -- регистрация платежа. Вставляет запись в \texttt{settlement\_payments}. Генерирует событие \texttt{payment:created}.
    
    \item \texttt{DELETE /api/groups/[id]/payments/[paymentId]} -- удаление платежа
\end{itemize}

\textbf{Приглашения} (\texttt{/api/groups/[id]/invites/*}):
\begin{itemize}
    \item \texttt{GET /api/groups/[id]/invites} -- список активных приглашений группы
    
    \item \texttt{POST /api/groups/[id]/invites} -- создание нового приглашения. Генерирует уникальный 6-символьный код и 48-символьный токен.
    
    \item \texttt{POST /api/invites/accept} -- принятие приглашения. Добавляет пользователя в \texttt{group\_members}, отмечает приглашение использованным.
\end{itemize}

\textbf{Обработка ошибок}:

Все API-маршруты используют единообразную обработку ошибок:
\cite{RFC7231}
\begin{itemize}
    \item 401 Unauthorized -- нет сессии или сессия истекла
    \item 403 Forbidden -- нет доступа к ресурсу
    \item 404 Not Found -- ресурс не найден
    \item 400 Bad Request -- невалидные данные
    \item 409 Conflict -- конфликт (например, email уже существует)
    \item 500 Internal Server Error -- ошибка сервера
\end{itemize}

Все ошибки возвращаются в формате \texttt{\{"error": "[Описание ошибки]"\}}.

\subsection{Клиентские компоненты}
React-компоненты организованы в папке \texttt{app/components/}

\textbf{AuthGuard.tsx} -- компонент-обёртка для проверки наличия сессии, при её отсутствии перенаправляет на страницу входа.

\textbf{GroupCard.tsx} -- карточка группы на главной странице. Отображает название, количество участников, общую сумму расходов.

\textbf{GroupView.tsx} -- главный компонент страницы группы.

\textbf{ExpenseList.tsx} -- список расходов группы. Отображает каждый расход с описанием, суммой, плательщиком, датой. Кнопки редактирования и удаления.

\textbf{ExpenseForm.tsx} -- форма добавления/редактирования расхода.

\textbf{SettlementList.tsx} -- список расчётов. Отображает кто кому и сколько должен. Кнопки для регистрации платежа. Показывает историю выполненных платежей.

\textbf{ParticipantList.tsx} -- список участников группы с отображением их балансов (переплата/долг).

\textbf{GroupInvitePanel.tsx} -- панель управления приглашениями. Показывает активные приглашения, позволяет создать новое, копировать код/ссылку в буфер обмена.

\textbf{JoinGroupForm.tsx} -- форма для присоединения к группе по коду приглашения.

\textbf{UserMenu.tsx} -- header приложения с отображением логотипа, имени и кнопкой выхода.

\begin{figure}[h]
    \centering
    \includegraphics[width=0.92\textwidth]{images/class_diagram.png}
    \caption{Диаграмма классов}
\end{figure}

\newpage
\section{Реализация}

\subsection{Система аутентификации}

Система аутентификации реализована в файле \texttt{app/lib/auth.ts}. Обрабатывает регистрацию, вход, управление сессиями и защиту паролей.

\textbf{Управление сессиями}

Сессии хранятся в таблице \texttt{sessions} БД. При входе создаётся новая сессия:

\begin{lstlisting}[language=JavaScript, basicstyle=\small\ttfamily]
export function createSession(userId: string): string {
  const sessionId = randomUUID();
  const expiresAt = new Date(
    Date.now() + 30 * 24 * 60 * 60 * 1000
  ).toISOString();  
  db.prepare(
    `INSERT INTO sessions (id, user_id, expires_at) 
     VALUES (?, ?, ?)`
  ).run(sessionId, userId, expiresAt);
  
  return sessionId;
}
\end{lstlisting}

Время жизни сессии -- 30 дней. При каждом запросе проверяется актуальность сессии:

\begin{lstlisting}[language=JavaScript, basicstyle=\small\ttfamily]
export function getSession(sessionId: string): 
  Session | null {
  const session = db.prepare(
    `SELECT * FROM sessions 
     WHERE id = ? AND expires_at > datetime('now')`
  ).get(sessionId);
  
  return session || null;
}
\end{lstlisting}

\textbf{Cookie-based аутентификация}

Идентификатор сессии хранится в HTTP-only cookie, откуда он и извлекается при необходимости валидировать сессию:

\begin{lstlisting}[language=JavaScript, basicstyle=\small\ttfamily]
export async function setSessionCookie(
  sessionId: string
): Promise<void> {
  const cookieStore = await cookies();
  cookieStore.set('tripf_session', sessionId, {
    httpOnly: true,
    sameSite: 'lax',
    maxAge: 30 * 24 * 60 * 60,
    path: '/',
  });
}
\end{lstlisting}

\subsection{Управление группами}

Функции для работы с группами находятся в \texttt{app/lib/groups.ts} и \newline \texttt{app/lib/invites.ts}.

\textbf{Система приглашений}

Для присоединения к группе используются два типа идентификаторов:

\begin{itemize}
    \item \textbf{Токен} -- для ссылок-приглашений вида \texttt{/invite/[token]}
    \item \textbf{Код} -- короткий код типа <<A3K9M2>> для ручного ввода
\end{itemize}

Генерация уникального кода (\texttt{app/lib/invites.ts}):

\begin{lstlisting}[language=JavaScript, basicstyle=\small\ttfamily]
const CODE_ALPHABET = 
  'ABCDEFGHIJKLMNOPQRSTUVWXYZ1234567890';
const CODE_LENGTH = 6;

function generateUniqueCode() {
  let code: string;
  do {
    code = '';
    for (let i = 0; i < CODE_LENGTH; i++) {
      const index = Math.floor(
        Math.random() * CODE_ALPHABET.length
      );
      code += CODE_ALPHABET[index];
    }
  } while (
    db.prepare(
      `SELECT 1 FROM group_invites WHERE code = ?`
    ).get(code)
  );
  return code;
}
\end{lstlisting}

Генерация приглашения:

\begin{lstlisting}[language=JavaScript, basicstyle=\small\ttfamily]
export function createGroupInvite(
  groupId: string, 
  createdBy: string, 
  expiresInHours: number = 72
): GroupInvite {
  const inviteId = randomUUID();
  const token = randomBytes(24).toString('hex');
  const code = generateUniqueCode();
  const expiresAt = new Date(
    Date.now() + expiresInHours * 60 * 60 * 1000
  );
  
  db.prepare(
    `INSERT INTO group_invites 
     (id, group_id, token, code, created_by, 
      expires_at) 
     VALUES (?, ?, ?, ?, ?, ?)`
  ).run(inviteId, groupId, token, code, createdBy, 
         expiresAt);
  
  return getInviteById(inviteId);
}
\end{lstlisting}

По умолчанию приглашение действует 72 часа.

\subsection{Учёт расходов}
\textbf{Создание расхода}

API-маршрут \texttt{POST /api/groups/[id]/expenses} принимает:

\begin{lstlisting}[language=JavaScript, basicstyle=\small\ttfamily]
{
  "description": "desc",
  "amount": 0000,
  "paidBy": "user-uuid",
  "splitType": "equal",
  "splits": [
    { "participantId": "user1-uuid", "amount": 0000 },
    { "participantId": "user2-uuid", "amount": 0000 },
    ...
  ]
}
\end{lstlisting}

Реализация в \texttt{app/api/groups/[id]/expenses/route.ts}:

\begin{lstlisting}[language=JavaScript, basicstyle=\small\ttfamily]
const expenseId = randomUUID();
const expenseDate = new Date().toISOString();

const insertExpense = db.transaction(() => {
  // |* Создаём расход *|
  db.prepare(
    `INSERT INTO expenses 
     (id, group_id, description, amount, 
      paid_by_user_id, split_type, date) 
     VALUES (?, ?, ?, ?, ?, ?, ?)`
  ).run(expenseId, groupId, description, amount, 
         paidBy, splitType, expenseDate);
  
  // |* Создаём разбивку *|
  const insertSplit = db.prepare(
    `INSERT INTO expense_splits 
     (expense_id, user_id, amount) 
     VALUES (?, ?, ?)`
  );
  
  for (const split of splits) {
    insertSplit.run(
      expenseId, 
      split.participantId, 
      split.amount
    );
  }
});

insertExpense();
\end{lstlisting}

Затем генерируем событие для обновления UI:

\begin{lstlisting}[language=JavaScript, basicstyle=\small\ttfamily]
emitGroupEvent(groupId, { 
  type: 'expense:created', 
  expenseId 
});
\end{lstlisting}

\subsection{Регистрация платежей}

Когда участники фактически производят переводы друг другу (через банк, наличными), они регистрируют это в приложении для пересчёта балансов.

\textbf{Создание платежа}

API-маршрут \texttt{POST /api/groups/[id]/payments}:

\begin{lstlisting}[language=JavaScript, basicstyle=\small\ttfamily]
export async function POST(request, { params }) {
  const userId = await getCurrentUserId();
  const { id: groupId } = await params;
  const { from, to, amount, note } = 
    await request.json();
  
  // |* Валидация *|
  if (!from || !to || !amount || amount <= 0) {
    return NextResponse.json(
      { error: 'Неверные данные' }, 
      { status: 400 }
    );
  }
  
  if (from === to) {
    return NextResponse.json(
      { error: 'Нельзя платить самому себе' }, 
      { status: 400 }
    );
  }
  
  const paymentId = randomUUID();
  const createdAt = new Date().toISOString();
  
  db.prepare(
    `INSERT INTO settlement_payments 
     (id, group_id, from_user_id, to_user_id, 
      amount, note, created_by, created_at) 
     VALUES (?, ?, ?, ?, ?, ?, ?, ?)`
  ).run(paymentId, groupId, from, to, amount, 
         note, userId, createdAt);
  
  // |* Генерируем событие *|
  emitGroupEvent(groupId, { 
    type: 'payment:created', 
    paymentId 
  });
  
  return NextResponse.json({ 
    payment: { id: paymentId, from, to, amount, 
               note, createdBy: userId, createdAt } 
  }, { status: 201 });
}
\end{lstlisting}

\subsection{Система событий}

Для синхронизации интерфейса при изменении данных используется простая реализация паттерна Observer в \texttt{app/lib/eventBus.ts}.

\textbf{Реализация EventBus}:

\begin{lstlisting}[language=JavaScript, basicstyle=\small\ttfamily]
type Listener = (payload: unknown) => void;

const listeners = 
  new Map<string, Set<Listener>>();

export function subscribeToGroup(
  groupId: string, 
  listener: Listener
): () => void {
  const groupListeners = 
    listeners.get(groupId) ?? new Set<Listener>();
  groupListeners.add(listener);
  listeners.set(groupId, groupListeners);
  
  // |* Возвращаем функцию отписки *|
  return () => {
    const current = listeners.get(groupId);
    if (!current) return;
    current.delete(listener);
    if (current.size === 0) {
      listeners.delete(groupId);
    }
  };
}

export function emitGroupEvent(
  groupId: string, 
  payload: unknown
): void {
  const groupListeners = listeners.get(groupId);
  if (!groupListeners) return;
  
  for (const listener of groupListeners) {
    try {
      listener(payload);
    } catch (error) {
      console.error('Event listener error:', error);
    }
  }
}
\end{lstlisting}

\textbf{Использование в компоненте}:

В \texttt{GroupView.tsx} компонент подписывается на события группы:

\begin{lstlisting}[language=JavaScript, basicstyle=\small\ttfamily]
useEffect(() => {
  const unsubscribe = subscribeToGroup(
    groupId, 
    (payload) => {
      // |* При любом изменении перезагружаем данные *|
      mutate(`/api/groups/${groupId}`);
      mutate(`/api/groups/${groupId}/expenses`);
      mutate(`/api/groups/${groupId}/payments`);
    }
  );
  
  return unsubscribe; // |* Отписка при размонтировании *|
}, [groupId]);
\end{lstlisting}

Функция \texttt{mutate} из SWR инвалидирует кэш и перезапрашивает данные с сервера. Это позволяет обновлять данные в UI при изменении данных на сервере, у всех пользователей в группе. Но такая система не сохранит события при перезагрузке сервера.

Для production-версии в идеале переписать на Redis Pub/Sub или WebSocket.

\newpage

\section{Тестирование}

Допустим есть 4 человека, которые хотять поехать в путешествие. Они регистрируются в приложении. Один из них создаёт группу и приглашает трёх других. Пользователи вводят коды приглашений и попадают в группу. Далее каждый вносит расходы в группу. Затем происходит автоматический расчёт долгов между участниками. После этого пользователи отмечают платежи как оплаченные и система автоматически рассчитывает балансы между участниками.

\begin{figure}[h]
    \centering
    \includegraphics[width=0.67\textwidth]{images/interface_example.png}
    \caption{Интерфейс приложения}
\end{figure}

\begin{figure}
    \centering
    \includegraphics[width=0.8\textwidth]{images/uasge_example.png}
    \caption{Сценарий использования}
\end{figure}

\newpage
\section{Дальнейшее развитие системы}

В текущей версии реализованы базовые функции для учёта групповых расходов. Для расширения возможностей приложения можно добавить возможность разделения расхода не только поровну, но и по процентам, суммам, долям.

Добавить мультивалютность, то есть возможность использования нескольких валют в одной группе. Сделать интеграцию с API курсов валют и пересчитывать суммы расходов в основную валюту группы.

Сделать категоризацию расходов (гостиницы, такси, питание, развлечения, ...).

Сделать дашборд с графиком расходов по категориям, участникам. Средний расход на человека и тому подобное.

Реализовать интеграцию с платёжными системами, например генерировать QR-код для переводов по СБП.

Возможность загружать фотографии чеков к расходам.

\newpage

\section{Заключение}

В ходе выполнения курсовой работы было разработано функциональное веб-приложение TripF для управления групповыми расходами.

Разработанное приложение решает реальную проблему справедливого разделения расходов в группах. Приложение готово к использованию в реальных поездках и групповых мероприятиях.

TripF имеет потенциал для дальнейшего развития: добавление мультивалютности, категорий расходов, статистики, интеграции с платёжными системами. Открытый исходный код позволяет сообществу участвовать в улучшении проекта.

Исходный код приложения доступен по ссылке: \url{https://github.com/arpefly/tripf}.

\newpage

\bibliographystyle{unsrt}
\bibliography{lib}

\newpage

\section{Приложение. Скриншоты интерфейса}

\begin{figure}[h]
    \centering
    \includegraphics[width=0.8\textwidth]{images/register_page.png}
    \caption{Форма регистрации}
\end{figure}

\begin{figure}[h]
    \centering
    \includegraphics[width=0.8\textwidth]{images/main_page.png}
    \caption{Главная страница}
\end{figure}

\begin{figure}[h]
    \centering
    \includegraphics[width=0.8\textwidth]{images/invite_pane.png}
    \caption{Панель создания приглашения}
\end{figure}

\begin{figure}[h]
    \centering
    \includegraphics[width=0.7\textwidth]{images/add_expense.png}
    \caption{Форма добавления расхода}
\end{figure}

\begin{figure}[h]
    \centering
    \includegraphics[width=0.7\textwidth]{images/settlement_list.png}
    \caption{Список расчётов}
\end{figure}

\end{document}
